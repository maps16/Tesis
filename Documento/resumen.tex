%%
%% resumen.tex
%%
%% Made by Carlos Calcaneo Roldan
%% Login   <calcaneo@z0.fisica.uson.mx>
%%
%% Started on  Thu Sep 27 09:56:40 2018 Carlos Calcaneo Roldan
%% Last update Time-stamp: <2019-oct-25.viernes 13:43:08 (oscar)>
%%

El modelo cosmológico actual asume que vivimos en un Universo ``críticamente lleno". Esto significa que la densidad  del Universo es exactamente aquella necesaria para mantenerlo abierto y sin curvatura. La manera mas exitosa para estudiar la cosmología en los últimos 50 años han sido las simulaciones cosmológicas, las cuales apuntan a un Universo dominado por una componente no material con aproximadamente un tercio de su contenido constituido por materia oscura $\Omega_0 \approx \Omega_\lambda + \Omega_M \approx 0.7+ 0.3$. La distribución de la materia en estas simulaciones nos da un mecanismo para contrastar estos valores de densidad con la observada.

%En este trabajo estudiamos la subestructura en simulaciones cosmológicas como posibles indicadores de los parámetros cosmológicos.

 { \blues En este trabajo se realizaron 8 simulaciones donde se variaron los parámetros de densidad de materia y de energía oscura, para estudiar los parámetros de masa, radio que contiene la mitad de la masa, radio de la velocidad circular maxima, la velocidad circular máxima y la dispersión de velocidades. Esto se realizo con la intención de usar los parámetros como indicadores del modelo cosmológico que se trataba.}



