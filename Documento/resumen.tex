%%
%% resumen.tex
%%
%% Made by Carlos Calcaneo Roldan
%% Login   <calcaneo@z0.fisica.uson.mx>
%%
%% Started on  Thu Sep 27 09:56:40 2018 Carlos Calcaneo Roldan
%% Last update Time-stamp: <2019-oct-25.viernes 13:43:08 (oscar)>
%%

El modelo cosmológico actual asume que vivimos en un Universo ``críticamente lleno". Esto significa que la densidad  del Universo es exactamente aquella necesaria para mantenerlo abierto y sin curvatura. La manera mas exitosa para estudiar la cosmología en los últimos 50 años han sido las simulaciones cosmológicas, las cuales apuntan a un Universo dominado por una componente no material con aproximadamente un tercio de su contenido constituido por materia oscura $\Omega_0 \approx \Omega_\lambda + \Omega_M \approx 0.7+ 0.3$. La distribución de la materia en estas simulaciones nos da un mecanismo para contrastar estos valores de densidad con la observada.

En este trabajo estudiamos la subestructura en simulaciones cosmológicas como posibles indicadores de los parámetros cosmológicos.












%==================================================================

%Analizamos una simulación de las órbitas de las estrellas del centro Galáctico, con el propósito de describir el movimiento de aquellos cuerpos para los cuales se conocen órbitas completas con mayor precisión. El objetivo es tener una mejor comprensión de las partes internas del potencial Galáctico y usar esto datos orbitales para cuantificar la cantidad y distribución de materia oscura en esta región. La simulación supone que el espacio-tiempo alrededor del agujero negro central de la galaxia puede ser modelado por la métrica de Schwartzchild, mientras que las interacciones estelares se aproximan de manera clásica. Modelamos el objeto central como un agujero negro con masa $4.31\times 10^6M_{\odot}$ ,  arreglamos el Galáctico distancia central a $R = 8.33kpc$ e incluye cinco estrellas en órbita (simulando los movimientos de S1, S2, S8, S12, S13, S14, S33), todos los cuales tienen masas aproximadamente de $20M_{\odot}$,excepto s2, que tiene un masa de $10M_{\odot}$. Nuestro método nos ha permitido reproducir algunas de las principales características orbitales. encontrado en la literatura y para predecir un cambio de perhelion, particularmente en S2, que es el más órbita completamente observada.


%==================================================================

%En este trabajo presentamos el desarrollo del problema de los tres cuerpos restringido, que consiste en tener dos cuerpos de masa grande en reposo y un tercer cuerpo de masa despreciable, para el cual estudiamos sus movimientos debido a la interacción gravitacional. En particular analizamos el sistema Tierra-Luna, aunque este modelo se puede emplear para explicar un gran n\'umero de sistemas.
%Para el análisis, consideramos el sistema desde un marco de referencia no inercial y construimos una {\it fuerza efectiva}, la cual contiene el término de fuerza gravitacional y dos términos que aparecen como una contribución debida al sistema de referencia no inercial: la fuerza centr\'ifuga  y la fuerza de Coriolis. Esta fuerza efectiva la podemos denotar como el gradiente de un potencial efectivo, mediante el cual obtenemos los puntos de Lagrange que denotamos $L1$, $L2$, $L3$, $L4$ y $L5$. Al analizar el movimiento del tercer cuerpo cuando sufre una perturbación en la posición de cada uno de estos puntos podemos asegurar que las \'orbitas alrededor de  $L1$, $L2$ y $L3$ son inestables, mientras que alrededor de $L4$ y $L5$ las \'orbitas son estables.
