%%%%%%%%%%%%%%%%% Begin
%%%%% Este códico es especial para agregar secciones en capítulos sin número, se usa en la introducción y en las conclusiones. 
\renewcommand{\altname}{Conclusiones}
\lhead[\fancyplain{}{}]%
      {\fancyplain{}{\bfseries \altname}}
\addchap{\altname}
%%%%%%%%%%%%%%%%% End
%%%%%%%%%%%%%%%%% Empezar a escribir abajo de esta línea. 
En este trabajo estudiamos el problema de tres cuerpos restringido y los puntos Lagrangianos del sistema Tierra\--Luna. Al adoptar un sistema de referencia no i%%%%% Este código es especial para agregar secciones en capítulos sin número, se usa en la introducción y en las conclusiones. Se pueden agregar tantas secciones como sea necesario, solo copiar las lineas entre los 
% (desde Begin hasta End). 
%\renewcommand{\altname}{Puntos L1,L2 y L3}%%%%% Solo necesitas %editar esta línea con el título de la sección
%\lhead[\fancyplain{}{}]%
%      {\fancyplain{}{\bfseries \altname}}
%\addsec{\altname}
%%%%%%%%%%%%%%%%% End
 

%%%%%%%%%%%%%%%%%%%%%%%%%%% Begin
%%%%%%%%%%%%%% No escribir  después de estas líneas. 
%%%%%%%%%%%%%% Este texto debe permanecer al final de este archivo!!!
\lhead[\fancyplain{}{}]%
      {\fancyplain{}{\bfseries\rightmark}}
%%%%%%%%%%%%%%%%%%%%%%%%%%% End
