%%
%% cap4.tex
%% 
%% Made by Carlos Calcaneo Roldan
%% Login   <calcaneo@jogrant>
%% 
%% Started on  Mon Jul 22 15:03:48 2019 Carlos Calcaneo Roldan
%% Last update Time-stamp: <2019-oct-28.lunes 10:06:50 (oscar)>
%%
\chapter{Función de amplificación para halos de materia oscura}
\label{cap:1}
\setcounter{equation}{0}

%%%%%%%% EL TExto Comienza abajo de aquí!

{\red Como antes, ¿De que se trata este capítulo? Ver ejemplo de cap1.}


\section{\red Modelos de lente extendida}



\subsection{Caso SIS}

\begin{equation} \label{eq:smd}
\Sigma(\xi)=\frac{\sigma_{v}^{2}}{2\xi}
\end{equation}

donde $\xi$ es la coordenada radial en el plano de la lente y $\sigma_{v}$ es la velocidad de dispersi\'on de los constituyentes del Halo. Asi como lo indica (citar a Schneider) este modelo fue motivado para ajustarse a las curvas de rotaci\'on de las galaxias espirales para valores de $\xi >> 1$ pero no predice bien el comportamiento para valores de $\xi$ pequeños ya que el modelo diverge en $\xi = 0$.


\subsection{Caso NFW}

