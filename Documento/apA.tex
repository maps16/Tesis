\appendix
\setcounter{chapter}{0}
\renewcommand{\chaptername}{Apéndice}
%\renewcommand{\thechapter}{\arabic{section}.\arabic{ind}}
\renewcommand{\theequation}{\Alph{chapter}.\arabic{section}.\arabic{equation}}
\addcontentsline{toc}{chapter}{\numberline{}Apéndice}
\setcounter{equation}{0}
\setcounter{figure}{0}
%%%%%%%%EL TExto Comienza abajo de aquí! 
\chapter{Realización de la Simulación} \label{appendix:Ser-Sim-INFO}
Veremos un resumen del servidor donde corrimos las simulaciones, así como, las características que se usaron para correr GADGET-4.

\section{Características del Servidor}
Se utilizó el servidor del \emph{Área de Cómputo del departamento de Física} (ACF). A continuación, en la tabla \ref{tab:Carac-Server}, mostramos un resumen de las características del servidor que utilizamos. 

\begin{table}[H]
    
    \centering
	
    \begin{tabular}{|c|c|}

        \hline
        \textbf{Característica} & \textbf{Especificación} \\ \hline
        CPU & 2x INTEL Xeon Gold 5120 \\ 
         & (2.20Ghz 19Mb Caché 14C/28T) \\ \hline
        Núcleos & 28 procesadores \\ 
         & (56 núcleos) \\ \hline
        Memoria & 4 X 8 GB.(32GB) \\
         & RDIMM 2666MT/S \\ \hline
        Almacenamiento & 6 TB Útiles con redundancia \\
         & (Arreglo RAID de discos duros de 2TB 7200K RPM S-ATA) \\ \hline

    \end{tabular}
	
    \caption{Se muestra un resumen de las características donde se realizaron las simulaciones.}

    \label{tab:Carac-Server}

\end{table}

\section{Características de las corridas de las Simulaciones}

La tabla \ref{tab:Comp-Gadget-4} muestra los parámetros que se usaron para compilar GADGET-4, con el cual activamos las diferentes herramientas que tiene. Mientras que la tabla \ref{tab:Corrida-Gadget-4} mostramos algunos de los parámetros que usamos para correr las simulaciones, las cuales se corrieron en paralelo con MPICH usando 54 de los 56 núcleos disponibles.

\begin{table}[H]

    \centering

    \begin{tabular}{|c|c|}
    	
        \hline
        \multicolumn{2}{|c|}{\textbf{Parámetro}} \\ \hline
        ASMT = 2.0 & IDS\_32BITS \\ \hline
        NTYPES = 2.0 & SUBFIND \\ \hline
        CREATE\_GRID & LEAN \\ \hline
        PERIODIC & SUBFIND\_HBT \\ \hline
        DOUBLEPRECISION = 2.0 & MERGERTREE \\ \hline 
        PMGRID = 384.0 & SELFGRAVITY \\ \hline
        FOF & NGENIC \\ \hline
        POSITION\_IN\_32BIT & TREEPM\_NOTIMESPLIT \\ \hline
        NSOFTCLASSES = 1.0 & NGGENIC\_2LPT \\ \hline
        \multicolumn{2}{|c|}{RANDOMIZE\_DOMAINCENTER }  \\ \hline
    	
    \end{tabular}

    \caption{Se muestra los parámetros de compilación de GADGET-4}

    \label{tab:Comp-Gadget-4}

\end{table}

\begin{table}[H]
    \begin{tabular}{|c|c|}
    \hline
    \multicolumn{2}{|c|}{\textbf{Parámetro}} \\ \hline
    MaxMemSize = 600 & TimeBegin = 0.015625 ($z=63$) \\ \hline
    Hubble = 100 & BoxSize = 50 \\ \hline
    UnitLength\_in\_cm = 3.085678e+24 & UnitMass\_in\_g = 1.989e+43 \\ \hline
    TimeMax = 1 ($z=0$) & HubbleParam = 0.678 \\ \hline
    NSample = 256 & Seed = 123456 \\ \hline
    UnitVelocity\_in\_cm\_per\_s = 100,000 & DesLinkNgb = 20 \\ \hline
    \end{tabular}
    \caption{Se muestran algunos los parámetros de corrida de las simulaciones con GADGET-4.}
    \label{tab:Corrida-Gadget-4}
\end{table}