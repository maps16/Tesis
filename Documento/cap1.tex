% !TeX root    = /home/martin/Documentos/Tesis/Documento/main.tex
%% cap1.tex
%%
%% Made by Carlos Calcaneo Roldan
%% Login   <calcaneo@jogrant>
%%
%% Started on  Mon Jul 22 15:01:12 2019 Carlos Calcaneo Roldan
%% Last update Time-stamp: <2021-ene-06.miércoles 12:27:54 (Oscar)>
%%

\chapter{Modelo Cosmológico}   %Introducción: Materia Oscura
\label{Chp:MatOsc}
\setcounter{equation}{0}

%%%%%%%%El Texto Comienza abajo de aquí!

%============================================================================================
%%===============================  NEW SECTION  =============================================
%============================================================================================
\setcounter{equation}{0}

Existe una gran cantidad de evidencia de que el Universo esta compuesto de una ``\textit{materia oscura}'' no luminosa y este material no es la materia bariónica habitual de la vida cotidiana (protones, electrones, neutrones, etc.) sino alguna  partícula con propiedades desconocidas. Determinar la naturaleza de la materia oscura es uno de los mas importantes  problemas sin resolver en la cosmología moderna.

Muchas escalas se han probado en la busca de evidencia de materia oscura: desde la escala cosmológica o la escala global del Universo  hasta la escala local de las galaxias. Recientemente, el segundo de estos métodos se encontró como el mas favorable debido a que es relativamente sencillo extraer información de la dinámica de sistemas cercanos. Experimentos en la escala cosmológica (e.g. WMAP \cite{ 2013ApJS..208...20B}, Misión Planck \cite{2020A&A...641A...1P}, o The Supernova Cosmology Project \cite{1999ApJ...517..565P}) hacen posible medidas detalladas de muchos parámetros cosmológicos. %De hecho, no es raro escuchar a muchos cosmólogos  decir que vivimos en ``La era de la precisión de la cosmología''.

%============================================================================================
%============================================================================================
%============================================================================================
\section{Materia y Energía en el Universo}

La cantidad y composición de materia y energía en el Universo es de fundamental importancia en la cosmología. Por simplicidad, todas las formas de materia y energía se pueden escribir como la fracción de la densidad de energía crítica:

\begin{equation*}
    \Omega_o \equiv \frac{\rho_{tot}}{\rho_o} = \Omega_{rad} + \Omega_M + \Omega_\Lambda
    \label{eq:fracDensEnerCrit}
\end{equation*}
donde los subíndices ``\textit{o}" denotan el valor de la época actual, $\rho_o =$ 3H$^2_o$/$8\pi$G $\approx$ 1.88h$^2$gcm$^{-3}$ $\approx$ 278h$^2$M$_\odot$kpc$^{-3}$ (en estas expresiones H$_o=$1000hkms$^{-1}$Mpc$^{-1}$ es la constante de Hubble y h es un numero en el rango de 0.5 a 1 - donde h$\sim$0.67 es el valor mas aceptado \cite{2013PASA...30...52C}). En la presente discusión descompondremos la densidad de materia/energía en tres componentes: la fracción aportada por la radiación (o especies relativistas) $\Omega_{rad}$, la componente de materia $ \Omega_{M} $ y una contribución suave $\Omega_\Lambda$. (No hay razón a priori para considerar solo estas componentes, esta elección trata de reflejar los valores medidos actuales, donde la materia y la radiación son componentes evidentes).

Una de las observaciones mejor definidas en la cosmología (con una precisión de 0.05$\%$) es que la radiación cósmica de fondo radia como cuerpo negro de temperatura T$_o=$2.7277 K. Esto significa que la contribución de los fotones a la densidad total de energía del Universo puede calcularse con precisión a ser $\Omega_\gamma$h$^2= 2.48\times 10^{-5}$. Si los neutrinos no tienen masa o son muy ligeros, entonces su densidad de energía también es muy conocida porque está relacionada con la de los fotones $\Omega_\nu$h$^2= 6\times \frac{7}{8}(4/11)^{\frac{4}{3}} \Omega_\gamma$ (considerando que hay 6 especies de neutrinos). La nucleosíntesis del Big Bang (BBN) restringe la cantidad de especies relativistas adicionales  menos que se hayan producido después de la época de BBN \cite{1999PhRvL..82.4176B}.

Debido a que la contribución de la radiación a la densidad total de energía en el Universo es pequeña, podemos continuar la discusión tomando en cuenta solamente las otras dos componentes $\Omega_M$ y $\Omega_\Lambda$

La temperatura de la Radiación Cósmica de Fondo (CMB de sus siglas en Ingles) es casi isotrópico en  todo el cielo. Esta es evidencia de que el Universo inició en un estado de densidad infinita. Pero, a una menor escala de prueba, el CMB presenta una anisotropía en la temperatura de $\Delta$T/T $\approx 10^{-5}$. Estas fluctuaciones se pueden utilizar para determinar el valor de $\Omega_o$.

En el estado caliente y denso del Universo temprano todas las partículas estaban ligadas. Esto incluye fotones y bariones (en general todas las partículas del modelo estándar). Conforme el Universo se enfrió, llegó un punto donde los fotones se dispersaron de los bariones. El CMB que observamos esta compuesto de fotones que provienen de la superficie de esta última dispersión. A medida que los bariones se acumulaban en los pozos de potencial de materia oscura, la presión de los fotones que actuó como una fuerza de restauración y esto resulto en una oscilación acústica impulsadas por la gravedad. Estas oscilaciones se pueden descomponer en sus modos de Fourier, las amplitudes multipolares $C_{lm}$ de la anisotropía del CMB están determinados por aquellos modos con $k\sim lH_o /2$. La última dispersión ocurrió por un pequeño periodo de tiempo, esto hace al CMB una foto del Universo en el momento de la última dispersión donde cada modo se puede ``ver'' en una fase bien definida de su oscilación. Los modos atrapados en la máxima compresión o rarefacción conducen a la anisotropía de la temperatura más grande, que resulta en una serie de picos acústicos. La posición del primer pico depende del valor de $\Omega_o$ (De hecho: $l \approx 200/\sqrt{\Omega_o}$) \cite{1999ASPC..165..431T}. Observaciones recientes del CMB \cite{ 2013ApJS..208...20B, 2020A&A...641A...1P}, permiten restringir la ubicación del prime pico ($l = 200 \pm 8$ \cite{2001ApJ...549..669H}) que a su vez fija el valor de la densidad total de materia/energía del Universo: $\Omega_o = 1 \pm 0.1$.

La abundancia primitiva predicha de $^4$He ($\approx 25\%$) fue el primer éxito del BBN. La consistencia de las predicciones de BBN  para las abundancia de estos elementos ligeros (D, $^3$He, $^4$He y $^7$Li) con sus abundancias primitivas inferidas ha sido una prueba importante del modelo Big-Bang en los primeros tiempos. De todos los elementos ligeros, el deuterio proporciona la mejor medición de la densidad de bariones en el Universo. Esto se debe a que la  abundancia primitiva de deuterio es más sensible a la densidad bariónica ($\propto 1/\rho_{baryon}^{1.7} $) y su evolución desde el Big-Bang es simple (las estrellas solo la destruyen).

Mediciones locales, donde alrededor de la mitad del material proviene de las estrellas, no refleja directamente las abundancias primordiales. Las lineas de Lyman del deuterio en el espectro de absorción de tres cuásares con alto corrimiento al rojo ($z>2$) han permitido una determinación precisa de las abundancias primordiales del deuterio $\rho_D/\rho_H = \left( 3.0 \pm 0.2 \right)\times 10^{-5 }$\cite{2001ApJ...552L...1B}. Esto nos conduce al valor de la densidad de bariones del Universo de $\Omega_{baryon}h^2 = 0.02 \pm 0.002$ \cite{2001ApJ...552L...1B}. Mediciones actuales del CMB también proveen un limite de $\Omega_{baryon}h^2 > 0.019$ \cite{2001ApJ...549..669H} que es consistente con los resultados anteriores.

%Una conclusión definitiva para el valor de $\Omega_o$ no esta muy lejos, esto sera resuelto por NASA's MAP y los satélites ESA's Plank. Podrán mapear el CMB en todo el cielo con una resolución $\left( \approx 0.1 ^{\circ} \right)$.

Otra forma para medir la densidad de masa/energía total, es a través del parámetro de desaceleración.
\begin{equation*}
    q_o \equiv -\frac{ ( \ddot{R}/R )_o }{H^2_o} = \frac{\Omega_o}{2}\left[ 1 + 3p_o/\rho_o \right]
\end{equation*}
el cuál cuantifica la desaceleración de la expansión debido a la gravedad.

Este método depende de un conocimiento preciso de la distancia luminosa($d_L$) para un objeto que, con un corrimiento al rojo bajo $z$, está relacionado con $q_o$ mediante
\begin{equation}
    d_LH_o = z +z^2(1-q_o)/2 +\mathcal{O}(z^3)
    \label{eq:dlho}
\end{equation}
por lo tanto, la precisión de las mediciones del flujo, $\mathcal{F}$, de objetos con luminosidad conocida, $L$, se puede usar para determinar $q_o$. (La distancia luminosa de un objeto se puede inferir de la ley del cuadrado inverso $d_L \equiv \sqrt{ L/4\pi\mathcal{F}}$ ).

Dos grupos (Supernova Cosmology Project y the High-z Supernova Team) usan mediciones precisas del flujo de objetos con luminosidad bien definida (Supernova tipo Ia - SNeIa) para concluir que la expansión del Universo se esta acelerando en lugar de desacelerarse, es decir $q_o < 0$ \cite{1998Natur.391...51P, 1998ApJ...507...46S}. Esto implica que gran parte de la energía en el Universo es una componente desconocida que presión negativa. La explicación mas popular para esta nueva componente es la existencia de una constante cosmológica, $\Omega_{\lambda} \neq 0$.% (Although we are warned - see e.g. Ref.[133] - that Eq.(1.1.1) is not accurate enough at the redshifts of the SNela, both groups actually compute $d_L$ as a function of $\Omega_M$ and $\Omega_{\lambda}$ therefore some modeling has been introduced.)

Por lo tanto, combinando observaciones modernas del CMB \cite{ 2013ApJS..208...20B, 2020A&A...641A...1P } , con recientes observaciones de supernovas \cite{1999ApJ...517..565P, 2006ApJ...644....1C}, el paradigma favorecido actualmente es un Universo en donde el contenido de materia corresponde aproximadamente a un tercio de la densidad total, es decir $\Omega_M \approx 0.3$ y existe una componente extra suave de energía oscura que contribuye $\Omega_{\lambda} \approx 0.7$. Actualmente se esta debatiendo la realidad y la naturaleza física de esta componente, pero por lo general es aceptada.

También es claro que el valor pequeño para la fracción de bariones en el Universo sugiere que la mayor parte de la materia ($\sim 90\%$) tiene que ser una forma de algún material no bariónico desconocido.


%\cite{2013ApJS..208...20B}
%\cite{2020A&A...641A...1P}
%============================================================================================
%============================================================================================
%============================================================================================
% \newpage
\section{Evidencia astrofísica de la Materia Oscura}
En 1932, Oort analizo números y velocidades de estrellas cercanas al sol y concluyó que la cantidad de materia gravitante implicada por estas velocidades era 30\% a 50\% superior a la que debía por las estrellas visibles. Después, en 1993, Zwicky concluyó que la velocidad de dispersión en cúmulos ricos de galaxias requieren de 10 a 100 veces mas masa para mantenerlos unidos de lo que podrían explicar las galaxias luminosas mismas.

No es simple concluir de estos ejemplos que la evidencia de alguna materia exótica, pero si ilustran como la dinámica de las estrellas, galaxias y cúmulos sirven como una sonda para el contenido de materia en el Universo.

%Porque reúnen material de una región grande del espacio, ricos cúmulo ricos proveen una buena muestra de materia en el Universo (los cúmulos típicamente se forman de perturbaciones de densidad con un tamaño comóvil de alrededor 10 Mpc). La evidencia de materia oscura en cúmulos ha sido tradicionalmente inferido de la aplicación del teorema del virial a movimientos de galaxias en los cúmulos. Porque el teorema del virial relaciona la energía cinética total (T) y energía potencial total(U) de un objeto ligado, es posible de extraer la masa total de un cúmulo asumiendo que el sistema esta en equilibrio.

%Existen diversas formas independientes de ``pesar'' cúmulos. Mapas de rayos-X y perfiles de temperatura son suficientes para estimar la profundidad del pozo gravitacional que confina el gas caliente emisor de rayos-X. Otro método esta basado en la detección de una gran número de galaxias de fondo muy tenues cuyas formas están distorsionadas por los efectos de Light Bending debido al campo gravitacional de los cúmulos. Esta técnica tiene la ventaja de que ofrece una forma muy directa de información sobre la distribución de masa total, cualquiera sean sus características.

%Otro forma de estimar $\Omega_M$ a partir de cúmulos, esta basado en un inventario de materia acoplado con BBN. La mayoría de los bariones en cúmulos residen en el gas intracúmulo caliente emisor de rayos-X y no en las galaxias mismas[150,157], el problema se reduce a determinar la proporción de gas-masa total, $f_{gas}$. Observaciones recientes de cúmulos con alta luminosidad de rayos-X [48*] nos dan un valor para $f_{gas} = \left(0.059 ^{+0.27}_{-0.24} \right)h^{3/2}$. Si los cúmulos son una buen muestra de la materia del Universo, entonces $\Omega_{baryon}/\Omega_M = f_{gas}$ y la determinación de $\Omega_{baryon}$ a partir del BBN se puede usar para inferir $\Omega_M = 0.3h^{1/2}$.

%También hay evidencia circunstancial a estas escalas (cúmulos en adelante) de que $\Omega_M$ es significativamente mayor que $\Omega_{baryon}$. La cantidad de estructura observada en la actualidad en el Universo imponen fuertes restricciones en la naturaleza de la materia dentro de el. No hay modelo viable para la formación de estructura sin una componente significante de materia no bariónica. En un Universo unicamente de bariones, las perturbaciones de densidad solo crecen del desacoplo del tiempo en $z \sim 1000$, hasta que el Universo queda dominado por la curvatura en $z \sim \Omega_{baryon}^{-1} \sim 20$. El tamaño de las perturbaciones de densidad inferidas de la anisotropía del CMB no tendrían el tiempo suficiente de crecer y formar estructuras no lineales que se observan actualmente. Con materia oscura no bariónica, las perturbaciones empiezan a crecer en igual forma materia/radiación y continúan creciendo hasta la época actual (o casi) llevando a significativamente mas crecimiento y haciendo que la estructura a gran escala observada sea consistente con el tamaño de las perturbaciones de densidad del CMB.

Más pruebas de la existencia de materia oscura provienen de la dinámica de sistemas galácticos, las curvas de rotación de galaxias espirales o la velocidad dispersión de enanas esferoidales. No se observa suficiente materia luminosa en las galaxias para explicar la cinemática observada, es decir, cuando las velocidades de galaxias espirales son medidas, generalmente se observa que al medir la velocidad de rotación lejos del centro, esta incrementa hasta alcanzar un valor constante. Esto contrasta con lo que se espera de una distribución de materia que corresponde a la materia luminosa (estrellas + gas) en la galaxia.

La escala mas pequeña, en la cual existe evidencia de materia oscura son las enanas esferoidales del Grupo Local. Observaciones de las velocidades de dispersión en la mayoría de estos sistemas \cite{1988IAUS..130..409A}, han soportado por un largo tiempo que son dominadas por materia oscura con proporciones de masa/luz en los rangos de 10-100.

%============================================================================================
%============================================================================================
%============================================================================================
% \newpage
\section{Características de la Materia Oscura}

Dado que la mayor parte de la densidad de materia del Universo no interactúa con los campos electromagnéticos, solo se detecta por su interacción gravitacional y posiblemente por interacciones de fuerza débil. Sin embargo, los efectos de la materia oscura sobre la materia visible son espectaculares, a medida que los pozos de potencial gravitacional de la materia oscura canalizan a los bariones formando los lugares de nacimiento de las galaxias visibles. El estudio de la ubicación, tamaño y la historia de fusión de galaxias están conectadas al crecimiento de estructuras de materia oscura, conocidas como ``halos''.

Estudiar de los halos de materia oscura no es simple.  El consenso dice que un halo es un objeto ligado gravitacionalmente \cite{2011MNRAS.415.2293K}. Pero no existe un consenso en la definición de sus propiedades. Existen propiedades con las que podemos caracterizar estas estructuras, entre ellas están la masa, las velocidades y su tamaño. Sin embargo,cada una de estas cantidades puede referirse a distintos aspectos físicos de la colección de partículas a la que llamamos ``halos''

Para poder hablar del tamaño, ocupamos hablar del radio. Es complicado especificar donde termina un halo debido a que la función de la densidad decrece con el radio y no es acotada, por lo cual es necesario definir un limite. Un buen limite es considerar el radio que contiene la mitad de la masa. Con este podemos dar una idea del tamaño, asi como, de que tan denso es el halo. 

Otro limite que podemos utilizar es el radio donde se encuentra el pico de la curva de rotación. Este limite nos permite darnos una idea de la dinámica del halo y de la densidad.

% Pensar en este párrafo 
Para poder describir la dinámica de los halos, estudiamos las diferentes velocidades asociadas al sistema. Para determinar la dinámica, podemos usar: la velocidad maxima radial, la velocidad del centro de masa del halo o bien la velocidad de dispersión. La velocidad maxima radial es aquella que se asocia con el pico de la curva de rotación. La velocidad del centro de masa nos permite conocer la dinámica del sistema con respecto al resto de los halos y podría llamarse ``la velocidad del halo en la simulación'' . Con la velocidad de dispersión podemos describir la dinámica interna de los halos. 

El estudio con precisión la velocidades, podemos comparar los datos experimentales con información obtenida de observaciones astronómicas.

Dada la naturaleza de la materia oscura, la mejor forma identificarla es mediante los efectos gravitacionales, por lo que estudiar la masa y su distribución nos permite conocer que tipo de estructuras podemos encontrar, o bien, como la materia oscura domina sobre la materia ordinaria, podemos inferir que estructuras de materia ordinaria se forman debido a la materia oscura.

% En los recientes años hemos visto un gran progreso en el estudio y modelado de estructura en pequeña y gran escala del Universo mediante el uso de simulaciones numéricas.
%============================================================================================
%============================================================================================
%============================================================================================
%\section{Candidatos a Materia Oscura}
% Ahora que revisamos algunos de los argumentos en favor de la existencia de una componente no bariónica del total de materia en el Universo, podemos mover nuestra atención a la naturaleza de este material.
%
% Existen muchos candidatos para la materia oscura. Desde los axiones con masa $m=10^{-5}GeV \approx 9 \times 10 ^{-72} M_{\astrosun}$, hasta agujeros negros de masa $m=10^4 M_{\astrosun}$. La propiedad básica de ser oscura no clarifica su naturaleza.  Pero existen una serie de esquemas de categorización, que son útiles para organizar los candidatos.
%
% Ya hemos analizado la primera distinción que se puede hacer: ¿Es bariónica  o no? Hemos examinando evidencia que gran parte de la materia oscura en el Universo es no bariónica. Más evidencia en contra materia oscura bariónica (al menos para tomar en cuenta la materia oscura en nuestra galaxia) proviene del estudio de micro-lensing hacia la Gran Nube de Magallanes (siglas en Ingles LMC) [83,3].
%
% Como hemos visto, hay argumentos convincentes en favor de una componente extra de materia que no es bariónica. Junto a los candidatos no bariónicos, un esquema de categorización importante es la ``Temperatura'' y la naturaleza de la interacción de las partículas de materia oscura. Los candidatos de materia oscura caliente (siglas en ingles HDM) son partículas que se mueven a velocidades relativistas al tiempo en el que las galaxias podrían comenzar a formarse. Los candidatos de materia oscura fría (siglas en ingles CDM) son partículas que se movían de manera no relativista en ese tiempo. Estudios de formación de galaxias pueden proveer pistas de que si la materia oscura es fría o caliente. CDM y HDM son partículas ``sin colisiones'' que interactúan muy débilmente entre si y los bariones. Otra posibilidad es materia oscura ``con colisiones'' en donde las partículas interactúan entre ellas (pero no con los bariones) mediante la fuerza fuerte.
%
% El candidato prototipo de HDM era un neutrino ligero. Si existe un neutrino de Dirac ligero ($m_{\nu}\lesssim 100$eV) existe, su densidad cosmológica sería de $\Omega_{\nu}h^2 \approx m_{\nu}/93$eV [74]. Simulaciones de N-Cuerpos de formación de estructura en un Universo dominado por HDM no reproducen la estructura observada [149] y por tanto se ha descartado.
%
% El principal candidato de CDM son los axiones y partículas masivas débilmente interactuantes (siglas en ingles WIMPs). Axiones son bosones ligeros sin spin que aparecen en modelos QCD. Búsqueda de laboratorio, enfriamiento de estelar y dinámica de la supernova 1987A restringen al axion a ser muy ligero (menos de unos pocos eV)[111]. La ventana donde estas partículas son candidatos viables de CDM se volvió pequeño, pero todavía existen rangos aceptables entre alrededor $10^{-5}$ y $10^{-2}$eV donde pasan todas las limitaciones observacionales (ver [14]). WIMPs son partículas estables que surgen como extensión del modelo estándar de interacciones electrodébiles. Los mas discutidos son los neutrinos pesados de cuarta generación de Dirac y Majorana y neutralino y neutrino en modelos de supersimetría. Las masas de WIMP, típicamente están en el rango de 100 GeV a unos pocos de eV y tienen interacciones con la materia ordinaria que son características de interacción débil.
%
% El candidato WIMP mas prometedor surge dentro del marco supersimetría mínima del modelo estándar (MSSM). Aunque estas partículas interactúan solamente débilmente con materia ordinaria, estas tienen una sección transversal distinta de cero para auto-interacción y un acoplamiento de interacción débil con la materia ordinaria. Mas adelante, esto abre la posibilidad de experimentos para la detección de materia oscura.
%
% Detección indirecta se puede lograr mediante la observación de fotones de alta energía (rayos-$\gamma$), los cuales se producen cuando partículas WIMP y sus antipartículas se encuentran y se aniquilan. Estos rayos-$\gamma$ se pueden producir con un espectro continuo de energía; cuando son producto del decaimiento de mesones $\pi^0$ producidos en chorros de aniquilaciones de WIMP, o pueden ser monocromáticos; cuando surgen como resultado de dos neutralinos aniquilándose directamente a dos fotones o un fotón y un bosón. Los rayos-$\gamma$ monocromáticos proporcionaría una excelente firma para la materia oscura si se llegara a observar en energías del orden de las masas WIMP. Aunque la forma detallada del encuentro para la aniquilación es difícil de describir concisamente, la interacción física (su probabilidad de dispersión intrínseca) esta contenida en $\sigma$, la sección eficaz para la aniquilación. Si $\sigma$ es conocida por una interacción de partículas a con partículas b, entonces podemos escribir
% \begin{equation*}
%     n_p = \left( n_a v \right) n_b\sigma
% \end{equation*}
% donde $n_p$ es el número de partículas producidas por la interacción por unidad de tiempo, $n_a$ el número de partículas ``a'', $n_b$ el número de partículas ``b'' y $v$ es la velocidad relativa entre ambas. Si consideramos solo aniquilaciones entre el mismo tipo de partícula y si en lugar de medir el numero de partículas interactuantes, solo tenemos una idea de la densidad de partícula $\tilde{n}$, entonces podemos reescribir la expresión como
% \begin{equation*}
%     \tilde{n}_p = \left( \sigma v \right)\tilde{n}^2
% \end{equation*}
% Si además consideramos la densidad física de la aniquilación de partículas $\rho$, llegamos a la expresión para el total de productos producidos después de la aniquilación de dos partículas supersimétricas:
% \begin{equation}
%     \tilde{n}_p = \left( v \sigma \right) \frac{\rho^2}{m^2_{\chi}}
%     \label{eq:densidadParticulasNP}
% \end{equation}
% Donde $m_{chi}$ representa la masa de una partícula supersimétrica. Las características de la partícula relevantes para la aniquilación están completamente contenidas en el parámetro $K = \left( v\sigma \right)/m_{\chi}^2$. La interacción de la sección transversal y la masa se calculan usando técnicas desarrolladas en QCD para la interacción de materia bariónica en donde los diagramas de Feynman se usan para describir todos los posibles procesos que llevan a esta interacción especifica [74]. En el caso particular del neutralino, le sección eficaz para la aniquilación se tiene que calcular para el continuo así como para los rayos-$\gamma$ monocromáticos (ver [13,138,12]). El rango de posibilidades para el parámetro $K$ es grande (desde $0.007 \times 10^{-30} $cm$^3$s$^{-1}$GeV$^{-2}$ hasta $150 \times 10^{-30} $cm$^3$s$^{-1}$GeV$^{-2}$) por lo tanto, en la ausencia de más restricciones, utilizaremos el valor medio de la velocidad promediada la sección transversal de $<\sigma v> = 10^{-30}$cm$^3$s$^{-1}$.
%
% Detección directa de WIMPs sería la última victoria para la teorías de materia oscura. Búsquedas de materia oscura dependen en el hecho de que estas partículas interactúan mediante fuerza débil con la materia ordinaria. En muchos de estos experimentos, la detección se logra midiendo los pequeños incrementos en temperatura causados dentro de un cristal puro cuando WIMPs llegan al núcleo y así este esquema de detección se puede utilizar para definir precisamente la naturaleza del material que impacto. Para asegurarse de que no hay calentamientos falsos debido a las interacciones con otras partículas (como los muones, que se encuentran todos los experimentos de detección de partículas), el cristal se aísla en laboratorios situados comúnmente localizados kilómetros debajo de la tierra (en secciones de minas sin uso).
%victoria
% A pesar de estos esfuerzos para aislar una detección clara, existe una posibilidad de que el aumento de temperatura del cristal se puede deber a otro cosa que no sean impactos de WIMP; cualquier detección dependerá en la confidencia en las estimaciones hechas para el fondo. ¿Como se puede eliminare esta confusión? Drukier, Frees y Spergel [43] han mostrado que, debido al movimiento de la tierra alrededor del sol (y el movimiento del sol en la galaxia), una señal producida por una WIMPs tendría modulación senoidal anual con picos en primavera. Por lo tanto, observar variaciones de temperatura del cristal aumenta durante el año puede traer una señal clara para la materia oscura.
%
% Recientemente, Spergel y Steinhardt [124] han propuesto la existencia de partículas de materia oscura fuertes auto-interacciones. Una característica clave de estas partículas (si llegan a reproducir características astrofísicas importantes) es que su camino libre medio esta en el rango de 1kpc-1Mpc en el radio solar. Esto implica que la relación de dispersión de su sección transversal y de su masa deben encontrarse en el mismo rango que los neutrones y protones ($\mathcal{O}$($10^{-23}$cm$^2$GeV$^{-1}$)), implicando que los candidatos naturales de este tipo de materia oscura son los hadrones - son partículas neutras y estables que interactúan a través de fuerzas fuertes entre ellas pero no con la materia ordinaria. Otra posibilidad es que haya un nuevo tipo de gluon que surge en modelos de supergravedad o supercuerdas. Porque estas partículas tienen una sección transversal grande para auto-interacción, son más similares a un gas que CDM estándar (es decir que interacciones fuertes ahora permiten la definición de \textit{temperatura en la materia oscura}) lo cual permite reproducir propiedades astrofísicas (como el número de halos en cúmulos) mas efectivamente que la CDM estándar.
%
% El progreso en la física fundamental puede que pronto nos diga si existe un nuevo tipo exótico de partícula y cuales son sus masas y secciones transversales. Calcular cuantos deben sobrevivir desde el primer milisegúndo y que tanto contribuyen a la densidad de materia oscura, en principio, es tan sencillo como calcular que tanto tiempo sobrevive el helio o el deuterio en los primeros minutos.
%
% No importa cual se la respuesta final al ``problema de la materia oscura'', esta claro que entender que la dinámica de sistemas astrofísicas es muy importante para completar una descripción del Universo.



%============================================================================================
%============================================================================================
%============================================================================================
% \section{Siguiendo la evolución de estructuras de CDM}
% 
% \cite{2009ApJ...707L.114G}
% 26    133
