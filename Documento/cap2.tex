%%
%% cap2.tex
%%
%% Made by Carlos Calcaneo Roldan
%% Login   <calcaneo@jogrant>
%%
%% Started on  Mon Jul 22 15:02:51 2019 Carlos Calcaneo Roldan
%% Last update Time-stamp: <2021-ene-06.miércoles 12:29:37 (oscar)>
%%

\chapter{Simulaciones Cosmológicas}
\label{chap:2 Sim}
\setcounter{equation}{0}
%%%%%%%%EL Texto Comienza abajo de aquí!

Las simulaciones cosmológicas son una herramienta esencial para el estudio del Universo. Es el único experimento con el contamos con el podemos reproducir su evolución. Las simulaciones numéricas nos permiten un estudio detallado de formación de estructura no lineal y nos permite hacer conexiones con un Universo simple con alto corrimiento al rojo, con uno complejo como el que se observa en la actualidad.

El trabajo que se realiza va desde el estudio del agrupamiento gravitacional no lineal de materia oscura, formación de grupos de galaxias, interacción de galaxias aisladas y la evolución de gas intergaláctico. Su desarrollo ha permitido un inmenso avance en estas áreas, el cual no podría ser posible  \cite{2001NewA....6...79S}.


Con la rápida evolución en el rendimiento de las computadoras y algoritmos numéricos, han nacido múltiples códigos que simulan la formación de estructura del Universo. Muchos de estos códigos se han hecho públicos y libres, lo que ha permitido que el estudio sea mas sencillo para nuevos investigadores o grupos de investigación \cite{2021MNRAS.506.2871S}. El futuro de las simulaciones numéricas esta en hacer mejoras en la precisión y en la fidelidad en la física de las técnicas de modelado.


\section{Grandes Simulaciones}

El desarrollo de códigos y el avance en el hardware de las maquinas modernas, has surgido grandes colaboraciones con la intención de realizar simulaciones cada vez mas grandes.

\subsubsection{Milenium Simulation}

\subsubsection{Eagle Simulation}
El proyecto EAGLE (Evolution and Assembly of GaLaxies and their Environments)  es una colaboración de The Virgo Consortium, la cual es una simulación hidrodinámica de gran escala de un Universo de $\Lambda$CDM, el cual tenia como objetivo entender como las galaxias se forman y evolucionan. La mas grande de las simulaciones se realizo con  6.8 billones de partículas, en un volumen de 100 Mpc por lado, conteniendo al menos 10,000 galaxias del tamaño de la vía láctea, la cual tardo mas de un mes y medio de tiempo de computo y una de la mas grandes supercomputadoras con 4000 núcleos de procesamiento usando una versión modificada del código de simulación GADGET-2 \cite{2015MNRAS.450.1937C, 2015MNRAS.446..521S}.

La simulación empieza en un Universo todavía muy uniforme (sin formación de estrellas o galaxias) usando parámetros motivos por las observaciones del satélite Plank y del CMB \cite{ 2013ApJS..208...20B, 2020A&A...641A...1P}.


\section{GADGET-4}
Los código GADGET han sido un de los mas utilizados en el estudió de formación de estructura y estudio de la materia oscura.



\subsubsection{---}




\subsubsection{---}
