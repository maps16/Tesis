%%
%% cap2.tex
%%
%% Made by Carlos Calcaneo Roldan
%% Login   <calcaneo@jogrant>
%%
%% Started on  Mon Jul 22 15:02:51 2019 Carlos Calcaneo Roldan
%% Last update Time-stamp: <2021-ene-06.miércoles 12:29:37 (Oscar)>
%%

\chapter{Simulaciones Cosmológicas}
\label{chap:2 Sim}
\setcounter{equation}{0}
%%%%%%%%EL Texto Comienza abajo de aquí!

\noindent Las simulaciones cosmológicas son una herramienta esencial para el estudio del Universo. Es el único experimento con el que contamos para reproducir su evolución. Las simulaciones numéricas nos permiten un estudio detallado de formación de estructura no lineal y nos permite hacer conexiones con un Universo simple con alto corrimiento al rojo, con uno complejo como el que se observa en la actualidad.

El trabajo que se realiza va desde el estudio del agrupamiento gravitacional no lineal de materia oscura, formación de grupos de galaxias, interacción de galaxias aisladas y la evolución de gas intergaláctico. Su desarrollo ha permitido un inmenso avance en estas áreas, el cual no era posible antes dado a las limitaciones que existían en el computó \cite{2001NewA....6...79S}.


Con la rápida evolución en el rendimiento de las computadoras y algoritmos numéricos, han nacido múltiples códigos que simulan la formación de estructura del Universo. Muchos de estos códigos se han hecho públicos y libres, lo que ha permitido que el estudio sea mas sencillo para nuevos investigadores o grupos de investigación \cite{2021MNRAS.506.2871S}. El futuro de las simulaciones numéricas esta en hacer mejoras en la precisión y en la fidelidad en la física de las técnicas de modelado.


\section{Grandes Simulaciones}

Gracias al desarrollo de códigos y el avance en el hardware de las maquinas modernas, has surgido grandes colaboraciones con la intención de realizar simulaciones cada vez mas grandes. Algunas de ellas son:

%==================================================================================================
%==================================================================================================
%==================================================================================================
\subsubsection{Millennium Simulation}

La Millennium Simulation fue una de las mas grandes simulaciones de N-Cuerpos realizada en el 2005. Esta contenía mas de 10 billones de partículas en una caja de 2 billones de años luz por lado. La simulación se realizo por la Virgo Consortium usando 512 procesadores localizados en el Instituto Max Planck para Astrofísica en Garching, Alemania. Tomo 28 días (~600 horas), consumiendo alrededor  de  343000 horas de tiempo de cpu \cite{2005Natur.435..629S}.

Esta fue la primera simulación parte de una serie de simulaciones relacionadas a volúmenes cosmológicos. En 2008 se realizo una segunda simulación con las misma cosmología, misma estructura de salida y misma cantidad de partículas, pero con una caja 5 veces mas pequeña, lo que permitió tener una resolución de masa mejor.

{\LARGE
IMAGEN PENDIENTE
}

%==================================================================================================
%==================================================================================================
%==================================================================================================
\subsubsection{Eagle Simulation}
El proyecto EAGLE (Evolution and Assembly of GaLaxies and their Environments)  es una colaboración de The Virgo Consortium, la cual es una simulación hidrodinámica de gran escala de un Universo de $\Lambda$CDM, el cual tenia como objetivo entender como las galaxias se forman y evolucionan. La mas grande de las simulaciones se realizo con  6.8 billones de partículas, en un volumen de 100 Mpc por lado, conteniendo al menos 10,000 galaxias del tamaño de la vía láctea, la cual tardo mas de un mes y medio de tiempo de computo y una de la mas grandes super-computadoras con 4000 núcleos de procesamiento usando una versión modificada del código de simulación GADGET-2 \cite{2015MNRAS.450.1937C, 2015MNRAS.446..521S}.

La simulación empieza en un Universo todavía muy uniforme (sin formación de estrellas o galaxias) usando parámetros motivados por las observaciones del satélite Plank y del CMB \cite{ 2013ApJS..208...20B, 2020A&A...641A...1P}. Algunos de los parámetros cruciales de la simulación son la densidad de materia oscura, la cual es responsable de la formación de estructura de materia bariónica, así como la constante cosmológica, responsable de la expansión acelerada del Universo.

{\LARGE
IMAGEN PENDIENTE
}
%==================================================================================================
%==================================================================================================
%==================================================================================================
\section{Simulaciones Cosmológicas}

Nuestro Universo se puede considerar un gas donde podemos decir que no hay colisiones entre las partículas que lo componen. Se asume esto ya que la distancia que hay entre las cosas que componen nuestro Universo se encuentran separadas grandes distancias y por tanto no hay gran probabilidad de que las cosas choquen. Por lo tanto, si queremos simular nuestro Universo, idealmente ocupamos resolver la ecuación de Boltzmann sin colisiones (CBE)

\begin{equation}
\frac{d f}{d t} \equiv \frac{\partial f}{\partial t} + \mathbf{v}\frac{\partial f}{\partial \mathbf{x}} + \frac{\partial \Phi}{\partial \mathbf{r}} \frac{\partial f}{\partial \mathbf{v}}
\label{eq:CBE}
\end{equation}

\noindent donde el potencial auto-consistente $\Phi$ es la solución a la ecuación de Poisson

\begin{equation}
\nabla^2\Phi(\mathbf{r},t) = 4\pi G \int f(\mathbf{r},\mathbf{v},t)d\mathbf{v}
\label{eq:PoissonSol}
\end{equation}

\noindent y $f(\mathbf{r},\mathbf{v},t)$ es la densidad de una partícula en el espacio fase.

Es muy complicado resolver este sistema de ecuaciones. La solución a la que se a llegado es construir códigos para simulaciones de N-Cuerpos. Existen grandes cantidades de códigos para simulaciones de N-Cuerpos, pero se diferencian en como realizan los cálculos para el movimiento gravitacional. Además de que siempre están buscando la forma de hacer los simuladores mas rápidos y eficientes.

%==================================================================================================
%==================================================================================================
%==================================================================================================

\section{GADGET-4}
Los código GADGET han sido un de los mas utilizados en el estudió de formación de estructura y estudio de la materia oscura en las ultimas décadas. Han existido varias iteraciones de los códigos GADGET, siendo GADGET-4 la mas reciente.

La motivación detrás de la nueva versión de GADGET-4
\cite{2021MNRAS.506.2871S}






\subsubsection{---}
Esencialmente para resolver el sistema de N-cuerpos, el potencial gravitacional que se esta usa para calcular el movimiento de las N partículas es el siguiente:

\begin{align}
    \Phi (\textbf{x}) = &- \sum_{j=1}^{N} \frac{m_j}{|\textbf{x}_j-\textbf{x}+\textbf{q}*_j| + |\epsilon(\textbf{x}_j-\textbf{x}+\textbf{q}*_j)|} \nonumber \\
    &+ \sum_{j=1}^{N} m_j \psi (\textbf{x}_j-\textbf{x}+\textbf{q}*_j)
\end{align}

donde la primer parte es el potencial gravitacional de newton con una corrección para considerar el radio de suavizado y el segundo termino en potencial se introduce como una corrección para el suavizado de imágenes distantes.


\subsubsection{---}
