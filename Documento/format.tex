%%%%%%%%%%%%%%%%%%%%%%%%%%%%%%%%%%%%%%%%%%%%%%%%%%%%%%%%%%%%%%%%%%%%%%%%%%
%     This is format.tex file needed for the dmathesis.cls file.  You    %
%  have to  put this file in the same directory with your thesis files.  %
%                Written by M. Imran 2001/06/18                          %
%                 No Copyright for this file                             %
%                 Save your time and enjoy it                            %
%                                                                        %
%%%%%%%%%%%%%%%%%%%%%%%%%%%%%%%%%%%%%%%%%%%%%%%%%%%%%%%%%%%%%%%%%%%%%%%%%%
%%%%%  Put packages you want to use here and 'fancyhdr' is required   %%%%
%%%%%%%%%%%%%%%%%%%%%%%%%%%%%%%%%%%%%%%%%%%%%%%%%%%%%%%%%%%%%%%%%%%%%%%%%%
\usepackage{fancyhdr}
\usepackage{epsfig}
\usepackage{cite}
\usepackage{graphicx}
\graphicspath{ {./images/}{./images/CanonRun/}{./images/CanonRun/Snapshot_DensityMap/}}
\usepackage{amsmath}
\usepackage{theorem}
\usepackage{amssymb}
\usepackage{latexsym}
\usepackage{epic}
\usepackage{ccicons}
\usepackage[backref=page]{hyperref}
\usepackage{unnumberedtotoc}
\usepackage{multirow}
\usepackage{subcaption}
\usepackage{mathrsfs}
\usepackage{ wasysym }
\usepackage{aas_macros}
%\usepackage[spanish]{babel}
% \usepackage{pgf}
\usepackage{float}
%%%%%%%%%%%%%%%%%%%%%%%%%%%%%%%%%%%%%%%%%%%%%%%%%%%%%%%%%%%%%%%%%%%%%%%%%%
%%%%%                 Set line spacing = 1.5 here                   %%%%%%
%%%%%%%%%%%%%%%%%%%%%%%%%%%%%%%%%%%%%%%%%%%%%%%%%%%%%%%%%%%%%%%%%%%%%%%%%%
\renewcommand{\baselinestretch}{1.5}
%%%%%%%%%%%%%%%%%%%%%%%%%%%%%%%%%%%%%%%%%%%%%%%%%%%%%%%%%%%%%%%%%%%%%%%%%%
%%%%%                      Your fancy heading                       %%%%%%
%%%%% For the final copy you need to remove '\bfseries\today' below %%%%%%
%%%%%%%%%%%%%%%%%%%%%%%%%%%%%%%%%%%%%%%%%%%%%%%%%%%%%%%%%%%%%%%%%%%%%%%%%%
\pagestyle{fancy}
\renewcommand{\chaptermark}[1]{\markright{\chaptername\ \thechapter.\ #1}}
\renewcommand{\sectionmark}[1]{\markright{\thesection.\ #1}{}}
\lhead[\fancyplain{}{}]%
      {\fancyplain{}{\bfseries\rightmark}}
\chead[\fancyplain{}{}]%
      {\fancyplain{}{}}
\rhead[\fancyplain{}{}]%
      {\fancyplain{}{\bfseries\thepage}}
\lfoot[\fancyplain{}{}]%
      {\fancyplain{}{}}
\cfoot[\fancyplain{}{}]%
      {\fancyplain{}{}}
\rfoot[\fancyplain{}{}]%
      {\fancyplain{}{}}%\bfseries\today}}
%%%%%%%%%%%%%%%%%%%%%%%%%%%%%%%%%%%%%%%%%%%%%%%%%%%%%%%%%%%%%%%%%%%%%%%%%%
%%%%%%%%%%%%Here you set the space between the main text%%%%%%%%%%%%%%%%%%
%%%%%%%%%%%%%%%%%%%and the start of the footnote%%%%%%%%%%%%%%%%%%%%%%%%%%
%%%%%%%%%%%%%%%%%%%%%%%%%%%%%%%%%%%%%%%%%%%%%%%%%%%%%%%%%%%%%%%%%%%%%%%%%%
\addtolength{\skip\footins}{5mm}
%%%%%%%%%%%%%%%%%%%%%%%%%%%%%%%%%%%%%%%%%%%%%%%%%%%%%%%%%%%%%%%%%%%%%%%%%%
%%%%%      Define new counter so you can have the equation           %%%%%
%%%%%    number 4.2.1a for example, this a gift from J.F.Blowey      %%%%%
%%%%%%%%%%%%%%%%%%%%%%%%%%%%%%%%%%%%%%%%%%%%%%%%%%%%%%%%%%%%%%%%%%%%%%%%%%
\newcounter{ind}

\def\eqlabon{

    \setcounter{ind}{\value{equation}}\addtocounter{ind}{1}
    \setcounter{equation}{0}
    \renewcommand{\theequation}{
        \arabic{chapter}%
        % .\arabic{section}%
        .\arabic{ind}\alph{equation}
    }
}

\def\eqlaboff{
    \renewcommand{\theequation}{
        \arabic{chapter}%
        % .\arabic{section}%
        .\arabic{equation}}
        \setcounter{equation}{\value{ind}
    }
}
%%%%%%%%%%%%%%%%%%%%%%%%%%%%%%%%%%%%%%%%%%%%%%%%%%%%%%%%%%%%%%%%%%%%%%%%%%
%%%%%%%%%%%%           New theorem you want to use              %%%%%%%%%%
%%%%%%%%%%%%%%%%%%%%%%%%%%%%%%%%%%%%%%%%%%%%%%%%%%%%%%%%%%%%%%%%%%%%%%%%%%
{\theorembodyfont{\rmfamily}\newtheorem{Pro}{{\textbf Proposition}}[section]}
{\theorembodyfont{\rmfamily}\newtheorem{The}{{\textbf Theorem}}[section]}
{\theorembodyfont{\rmfamily}\newtheorem{Def}[The]{{\textbf Definition}}}
{\theorembodyfont{\rmfamily}\newtheorem{Cor}[The]{{\textbf Corollary}}}
{\theorembodyfont{\rmfamily}\newtheorem{Lem}[The]{{\textbf Lemma}}}
{\theorembodyfont{\rmfamily}\newtheorem{Exp}{{\textbf Example}}[section]}
\def\remark{\textbf{Remark}:}
\def\remarks{\textbf{Remarks}:}
\def\bproof{\textbf{Proof}: }
\def\eproof{\hfill$\Box$}
%%%%%%%%%%%%%%%%%%%%%%%%%%%%%%%%%%%%%%%%%%%%%%%%%%%%%%%%%%%%%%%%%%%%%%%%%%
%%%%%%%    Bold font in math mode, a gift from J.F.Blowey       %%%%%%%%%%
%%%%%%%%%%%%%%%%%%%%%%%%%%%%%%%%%%%%%%%%%%%%%%%%%%%%%%%%%%%%%%%%%%%%%%%%%%
\def\bv#1{\mbox{\boldmath$#1$}}
%%%%%%%%%%%%%%%%%%%%%%%%%%%%%%%%%%%%%%%%%%%%%%%%%%%%%%%%%%%%%%%%%%%%%%%%%%
%%%%%%%        New symbol which is not defined in Latex         %%%%%%%%%%
%%%%%%%                 a gift from J.F.Blowey                  %%%%%%%%%%
%%%%%%%%%%%%%%%%%%%%%%%%%%%%%%%%%%%%%%%%%%%%%%%%%%%%%%%%%%%%%%%%%%%%%%%%%%
% The Mean INTegral
% to be used in displaystyle
\def\mint{\textstyle\mints\displaystyle}
% to be used in textstyle
\def\mints{\int\!\!\!\!\!\!{\rm-}\ }
%
% The Mean SUM
% to be used in displaystyle
\def\msum{\textstyle\msums\displaystyle}
% to be used in textstyle
\def\msums{\sum\!\!\!\!\!\!\!{\rm-}\ }
%%%%%%%%%%%%%%%%%%%%%%%%%%%%%%%%%%%%%%%%%%%%%%%%%%%%%%%%%%%%%%%%%%%%%%%%%%
%%%%%%%%%%            Define your short cut here              %%%%%%%%%%%%
%%%%%%%%%%%%%%%%%%%%%%%%%%%%%%%%%%%%%%%%%%%%%%%%%%%%%%%%%%%%%%%%%%%%%%%%%%
\def\poincare{Poincar\'e }
\def\holder{H\"older }


%%%%%%%%%%%%%%%%%%%%%%%%%%%%%%%%%%%%%%%%%%%%%%%%%%%%%%%%%%%%%%%%%%%%%%%
%%%
%%%   Octava parte del Pre�mbulo: Definiciones especiales
%%%
%%%%%%%%%%%%%%%%%%%%%%%%%%%%%%%%%%%%%%%%%%%%%%%%%%%%%%%%%%%%%%%%%%%%%%%
%%%%% BLOQUE K inicio.
\def\etal{{\em et al.}}
\newcommand{\eee}{\ensuremath{{\rm e}\,}}   %e en lugar de exp.
\newcommand{\dif}{\ensuremath{{\rm d}}}     %d derecha para las derivadas.
\newcommand{\rmG}{\ensuremath{{\rm G}}} %Constante gravitacional.
\newcommand{\Lcal}{\ensuremath{\mathscr{L}}} %Ele mayúscula
                                %caligráfica
\newcommand{\Ecal}{\ensuremath{\mathscr{E}}} %E mayúscula
                                %caligráfica
\newcommand{\Vcal}{\ensuremath{\mathscr{V}}} %V mayúscula
                                %caligráfica
\newcommand{\vLcal}{\ensuremath{\stackrel{\longrightarrow}{\mathscr{L}}}}
                                %Ele mayúscula caligráfica y vector
%%%%% BLOQUE K fin.


%para colores {BEGIN}
\usepackage{pstricks}
\newrgbcolor{darkgreen}{0 .5 0}
\newrgbcolor{blues}{0 0 .5}
\newrgbcolor{morado}{.3984 .1992 .9961}
\newrgbcolor{cafe}{.5391 .3594 .1797}
\newrgbcolor{cafeoscuro}{.4 .26 .13}
\newrgbcolor{darksienna}{0.24, 0.08, 0.08}
%para colores{END}