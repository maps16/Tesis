\setcounter{equation}{0}

%=========================================== Begin ===============================================
%   Este código es especial para agregar secciones en capítulos sin número, se usa en la  introducción y en las conclusiones.
%=================================================================================================
\newcommand{\altname}{Introducción}
\lhead[\fancyplain{}{}]%
      {\fancyplain{}{\bfseries \altname}}
\addchap{\altname}

El estudio de la Cosmología tiene una historia tan larga como la historia de la humanidad. Desde que tenemos registro las personas han volteado al cielo para tratar de describir su lugar en el Universo. En un principio fue por razones muy prácticas, la predicción de las estaciones del año y las épocas de lluvia para garantizar la cosecha. Sin embargo los seres humanos al ser tan capaces para encontrar patrones, quisieron entrelazar la regularidad de los astros con sus vidas, para encontrar explicación sobre los acontecimientos cotidianos. Así, en sus orígenes, la Cosmología esta muy entrelazada con la creencia y el mito.

Conforme fueron mejorando los instrumentos de medición, los seres humanos pudimos encontrar en la regularidad del Universo una confirmación de la Física que ocurría en el día a día. La ley de gravitación Universal es un primer ejemplo de una teoría Física de aplicación terrestre, pero que tiene alcance Universal, de ahí su importancia. Para mediados del siglo 20, la tecnología de observación de las estrellas era tan buena que podían empezarse a contrastar las ideas de la cosmología con la observación y confirmación en las estructuras que se veían en el cielo. {\blues Nace la Cosmología Física ya como una ciencia. }% y no como parte de la Filosofía o la especulación y el mito.

En la década de los 80, gracias al advenimiento de equipos de cómputo cada vez más poderosos, en particular la capacidad de enlazar varias computadoras para hacerlas funcionar como una sola, fue posible empezar a realizar simulaciones de N-Cuerpos, es decir colocar pequeños puntos con masa en un volumen y dejarlos evolucionar usando la ley de gravitación universal para describir su comportamiento.

Estas simulaciones han crecido enormemente desde sus orígenes y ahora es posible realizarlas en una computadora sencilla de escritorio, lo cual dice mucho del avance en el equipo de cómputo y de las técnicas en los algoritmos que resuelven las ecuaciones de movimiento para las partículas de la simulación.

En este trabajo resumimos los resultados de usar una serie de algoritmos, los cuales  podemos llamar colectivamente {\blues GADGET-4 \cite{2001NewA....6...79S,2021MNRAS.506.2871S}}, que están disponibles para hacer simulaciones cosmológicas. Mi trabajo principal consistió en aprender a ajustar los modelos necesarios y correr distintos escenarios Cosmológicos para corroborar que las simulaciones pueden servir para discernir el Universo en el que vivimos de una vasta posibilidad de modelos.

Después de un breve resumen sobre el modelo cosmológico en el capitulo 1, describimos las técnicas para simular partículas usando {\blues GADGET-4} en el capitulo 2. En el capítulo 3 presentamos un resumen de todas las simulaciones realizadas, mientras que en el capítulo 4 concluimos que efectivamente existen varios indicadores que nos permite distinguir un modelo cosmológico de otro. Esto además permitirá comparar con observaciones del espacio real, así como guiar el tipo de observaciones que se requieren para poder así acercarnos al modelo que mejor describe el Universo el que vivimos.


%===========================================  End ================================================

%======================== Empezar a escribir abajo de esta línea. ================================
%\textbf {Breaking News: LIGO just anounce they found the gravitational wave, February 11  2016 9:43 a.m. :) }






%=========================================== Begin ===============================================
%============================== No escribir  después de estas líneas. ============================
%==================== Este texto debe permanecer al final de este archivo!!! =====================
\lhead[\fancyplain{}{}]%
      {\fancyplain{}{\bfseries\rightmark}}
%===========================================  End ================================================
