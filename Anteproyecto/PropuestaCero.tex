%%
%% PropuestaCero.tex
%% 
%% Made by Carlos Calcaneo Roldan
%% Login   <calcaneo@billpots>
%% 
%% Started on  Thu Mar 19 09:22:26 2020 Carlos Calcaneo Roldan
%% Last update Time-stamp: <2020-Mar-19.Thursday 19:28:28 (calcaneo)>
%%
\documentclass[12pt,spanish]{article}

%La sigientes lineas son para podel escribir en español usando el
%teclaso y poder usar el corrector de idioma automático. 
%OJOJOJOJOJOJOJOJOJOJOJOJOJO:  
%También es necesario el spanish como opción de la clase.
%OJOJOJOJOJOJOJOJOJOJOJOJOJO:  
\usepackage[spanish]{babel}
\selectlanguage{spanish}
\spanishdecimal{.}
\usepackage[utf8]{inputenc}



\usepackage{helvet}  %Para usar tipo de letra helvEtica.
%\usepackage{courier} %Para usar tipo de letra courier
\renewcommand{\familydefault}{\sfdefault} %Cambia el tipo de letra
                                %default a helvEtica para todo el
                                %documento, inluyendo los textos que
                                %se generan de manera automAtica.
\usepackage{pstricks,pst-coil,pst-node}
\usepackage{diagbox}
\usepackage{epsfig,graphics,fancyvrb}
\usepackage{fancyhdr,lastpage}
\usepackage{amsmath,amssymb,amsfonts,mathrsfs}
\usepackage{enumerate}
\setlength{\topmargin}{-1mm}   %Separacion del margen superior al 
                               %inicio de la cabeza. 
\setlength{\footskip}{8mm}     %Separacion del pie del texto.
\setlength{\headsep}{3mm}      %Separacion de la cabeza del texto.
\setlength{\headheight}{6mm}   %Altura de la cabeza del texto.
%\addtolength{\voffset}{-1in}  %Margen superior total de la hoja.
% longitud de la cabeza y el pie fijados a la longitud del texto, 
% pueden tener su propia longitud.
%\renewcommand{\headwidth}{\textwidth}  % Notemos que \headwith controla
                                       % tanto cabeza como pie.
\setlength{\textwidth}{160mm}
\setlength{\textheight}{220mm}
\pagestyle{fancy}
\renewcommand{\thefootnote}{\fnsymbol{footnote}}
%\newcommand{\sen}{\ensuremath{{\rm sen}\,}}
\newcommand{\dif}{\ensuremath{{\rm d}}}
\newcommand{\etal}{\em et al.}
			
%\lhead{\tiny  Proyecto FOMIX}
%Para camiar sOlo el texto C. Calc\'aneo Rold\'an podriamos escribir:
%\chead{\tiny \fontfamily{phv}\selectfont C. Calc\'aneo Rold\'an}
%\chead{\tiny C. Calc\'aneo Rold\'an}
\newcommand{\mydate}{19 de Marzo de 2020}
\rhead{\tiny Propuesta de tema preliminar de investigación}
\rfoot{\thepage \mbox{\,} de \pageref{LastPage}}
\cfoot{}
\lfoot{\tiny \mydate}

\begin{document}
%\fontfamily{phv}\selectfont %Esta linea cambia a helvEtica el texto
                            %normal del documento. pero solo el texto,
                            %no cambia los textos que se generan
                            %automAticamente como ``Referencias'' o
                            %``Indice''... 
                            %{phv}{Adobe Helvetica}
                            %{pcr}{Adobe Courier}
\thispagestyle{empty}
\setcounter{page}{0}
\begin{center}
  {\Large Propuesta de tema preliminar de investigación\\
  ``{\it Encontrando amigos oscuros}''}
\end{center} 

\vspace{12em}
\begin{center}
  {\large Asociado a la LGAC del programa de maestría: Física matemática. \\}
\vspace{4em}
{\large Resumen:}
\end{center}
Presentamos una propuesta para el desarrollo de una investigación
asociada al estudio de halos de materia oscura usando técnicas de
aprendizaje de máquina (conocido técnicamente como {\it machine
  learning}). En esta propuesta se incorporará esta técnica a la búsqueda
de sobre-densidades en simulaciones de materia oscura para lo cual
tradicionalmente se usan otras técnicas.

\begin{flushright}
\scalebox{0.75}{
\includegraphics{/home/calcaneo/img/firmaKlKblue.png}
}\\
\vspace{-4ex}
Dr. Carlos Calc\'aneo Rold\'an\\
Academia de Física Teórica\\
Departamento de Física.\\
NAC Maestría en Ciencias \--- Física. 
\end{flushright}


\vspace{5em}
%\begin{center}
%     Referencia CONACyT: \\
%     {\bf CB\--2009\--01\_000000000128851}
%
%\end{center}

\vfill
\begin{flushright}
{\bf \mydate}
\end{flushright}
\newpage

\section*{Antecedentes}

Una gran parte del trabajo que desarrollamos en Cosmología en el grupo
de Física Fundamental se sustenta en los siguientes antecedentes que
hemos mencionado ya en otros momentos:

\begin{quote}

Existe un cúmulo de evidencia experimental y teórica
que nos hace  
confiar en el modelo cosmológico actual, mismo que  ha surgido de
la confluencia entre la cosmología, la astrofísica y la física
de partículas elementales. Las primeras describen el universo a gran
escala, mientras la última se encarga de estudiar el mundo
subatómico, su unión nos ha llevado a la realización de
experimentos que están en la frontera de las posibilidades humanas,
tanto materiales como intelectuales: LHC, WMAP, CDMS, LISA; por
mencionar sólo algunos de los más nombrados en la actualidad,
incluso en la prensa popular.

El paradigma actual de la formación del Universo Observable se 
basa en una serie de suposiciones sencillas: Momentos después del 
gran estallido existían pequeñas imperfecciones en la 
distribución del plasma materia/energía que formaba al Universo.
Al ir expandiéndose, el plasma se fue enfriando hasta dejar escapar 
a los fotones. Pequeñas imperfecciones en la distribución de
materia fueron 
expandidas exponencialmente por un proceso de inflación de tal 
manera que se convirtieron en grumos de materia que se 
identifican comúnmente como las semillas de las galaxias. Los 
pequeños grumos van acumulándose en objetos cada vez más 
grandes, a través de interacciones puramente 
gravitacionales,  hasta formar halos dentro de los cuales, de nuevo 
debido a la atracción gravitacional, se acumula materia 
bariónica hasta formar galaxias. Posteriormente las galaxias se 
agregan en cúmulos y así sucesivamente.

Para poder reproducir observaciones astrofísicas (como las 
curvas de velocidad circular de galaxias o la dispersión de 
velocidades dentro de un cúmulo de galaxias) ha sido necesario 
suponer que la mayoría de la materia contenida en los halos 
galácticos tiene características muy peculiares. Esta materia 
no puede corresponder a la materia bariónica que conocemos, 
su principal característica es que las partículas que forman 
esa materia sólo deben interactuar con las partículas de la 
materia ordinaria a través de la fuerza gravitacional. Como 
dicha materia no interactúa  termodinámicamente, no incrementa 
la temperatura de su distribución en el halo y por lo tanto no 
radia. Es por ello que se le ha denominado ``Materia Oscura''.

Se han propuesto muchos candidatos de materia oscura, los más 
aceptados pertenecen a una clase denominada ``Materia Oscura 
Fría'' (CDM, por sus siglas en inglés). 


\end{quote}

\section*{Justificación}

Los argumentos principales para continuar con el estudio de halos de
materia oscura también los hemos resaltado antes, ahora recordamos la
esencia de ellos:

\begin{quote}

  
Uno de los modelos de formación de estructura estudiados más 
intensamente se  basa en un universo dominado por CDM. Este modelo ha
sido explorado en gran detalle por casi 30
años a través de simulaciones y modelado semi-analítico. Una 
propiedad importante de las simulaciones es que los halos pequeños 
son los primeros en colapsarse y separarse de la expansión del 
Universo. Estos halos crecen lentamente al acretar
grumos más pequeños o rápidamente al fusionarse con otro halo de 
tamaño comparable. En otras palabras, la formación de la estructura 
es jerárquica: los objetos pequeños se forman primero y los más 
grandes después.

Una consecuencia importante de este paradigma cosmológico es que las
galaxias se forman al centro de los halos de materia  
oscura y que, como cerca del 80\% de la materia del halo es materia 
oscura, ésta domina las interacciones entre las galaxias. 
Así, estudiando estas interacciones idealizadas esperamos entender 
cuáles son los factores que influyen en la evolución de las
galaxias y en general la evolución de la estructura del Universo. 

La razón principal para estudiar las interacciones entre halos 
de materia oscura es que su dinámica nos permite entender la 
evolución de las galaxias dentro de cúmulos de galaxias o bien de 
satélites dentro de la Vía Láctea. Las galaxias se observan en 
diferentes ambientes, desde completamente aisladas en el 
campo hasta agrupadas densamente en un cúmulo de galaxias. Una 
parte importante de estudiar la evolución de galaxias es entender las 
transformaciones que sufren mientras orbitan dentro de este ambiente 
denso.

\end{quote}

\section*{Objetivo}

Tradicionalmente en la búsqueda de estructura en los halos de materia
oscura se utiliza un método conocido como {\it ``Friends of Friends''},
o amigos de amigos, en los que se recogen sobre-densidades usando
distancias características entre las partículas de una simulación.

En este trabajo nos proponemos aprender técnicas similares a las
propuestas por Yongseok y Ji-hoon \cite{mssm} o Liao, Guo et
al. \cite{hiker} y usar algoritmos novedosos de aprendizaje de
máquina, mejor conocido 
como {\it ``machine learning''} (ML de aquí en adelante) para
encontrar subestructuras en halos de materia oscura.

\section*{Metas}

\begin{itemize}
\item Completar un algoritmo de ML para la determinación de
  subestructuras de halos de materia oscura en una simulación
  cosmológica. 
\item Comparar el método de ML con los métodos tradicionales para
  encontrar estructuras.
\end{itemize}

\section*{Resultados esperados}

\begin{itemize}
\item Simulación cosmológica propia, o bien descarga de datos de
  alguna fuente confiable.
\item Algoritmo de ML para la detección de halos de materia oscura.
\item Catálogo de halos de subestructura de materia oscura.
\item Resumen del trabajo realizado en una disertación para obtener el
  grado de Maestría en ciencias. 
\end{itemize}

\section*{Infraestructura Disponible}

La investigación que nos proponemos realizar se apoya fuertemente
en el cómputo mediante cálculos numéricos y el desarrollo y
análisis de simulaciones. Gran parte del tiempo se invertirá en
mejorar y diseñar nuevos algoritmos por lo que es de importancia
fundamental la disponibilidad tanto de un buen equipo de cómputo
como de software y compiladores actualizados que
permitan la implementación y el desarrollo del trabajo.

Actualmente realizamos el trabajo numérico  en un
servidor intel con procesador Core 2 Quad Q9550, 8Gb de memoria RAM
DDRII y capacidad de disco duro combinada de 1Tb
y contamos con una impresora Láser Dell Multifuncional.

En lo que concierne al material bibliográfico, el Departamento de
Física cuenta con un acervo considerable, mismo que en su
oportunidad habrá de actualizarse. Entre otras revistas
de investigación se tiene acceso a las siguientes: The Physical
Review D, Physical Review Letters, The Astrophysical Journal, Journal
of Mathematical Physics, Revista Mexicana de Física, Revista
Mexicana de Astronomía y Astrofísica. Además se tiene acceso en
línea a la colección de revistas del IOP, AIP y de la APS.  

En cuanto a recursos de software contamos con los paquetes
Mathematica (versión 11.3), SuperMongo
(versión 2.4.36), sistema operativo Linux con compiladores y
graficadores de fuente abierta instalados. Realizamos el trabajo 
de edición en \LaTeX .

%%%%%%%%%%%%%%%%%%%%%%%%BIBLIOGRAFIA
\begin{thebibliography}{00}
\bibitem{mssm} Machine-assisted semi-simulation model (MSSM):
  estimating galactic baryonic properties from their dark matter using
  a machine trained on hydrodynamic simulations. J. Yongseok,
  K. Ji-hoon. {\it MNRAS}, {\bf 489}:3565,2019.
\bibitem{quijote} The Quijote simulations. Villaescusa-Navarro, F.,
  Hahn, C., Massara, E., et al.\ 2019, arXiv e-prints,
  arXiv:1909.05273
\bibitem{hiker} HIKER: a halo-finding method based on kernel-shift
  algorithm. Sun, S., Liao, S., Guo, Q., et al.\ 2019, arXiv e-prints,
  arXiv:1909.13301 
\bibitem{aar} Gravitational N-Body Simulations \--- Tools and
  Algorithms. S. J. Aarseth. {\it Cambridge University Press},
  Cambridge, UK, Primera edición, 2003. ISBN:0521432723.
\bibitem{bin01} Galactic Dynamics. J. Binney, S. Tremaine. {\it Princeton 
University Press}, Princeton, NJ, segunda edición,
  2008. ISBN:978-0691130279.
\bibitem{boy} Resolving Cosmic Structure Formation with the
  Millennium-II Simulation. M. Boylan-Kolchin, V. Springel,
  S. D. M. White, A. Jenkins, G. Lemson. {\it MNRAS}, {\bf
    398}:1150,2009. 
%\bibitem{cal} Galaxy destruction and diffuse light in
%  clusters. C. Calcaneo-Roldan, B. Moore, J. Bland-Hawthorn, D. malin,
%  E. M. Sadler. {\it MNRAS}, {\bf 314}:324, 2000.
%\bibitem{eke01} The evolution of X-ray clusters in a low density universe. 
%V. R. Eke, J. F. Navarro, C. S. Frenk. {\it ApJ}, {\bf 503}:569, 1998.
%\bibitem{eke02} The power spectrum dependenceof dark matter halo
%  concentrations. V. R. Eke, J. F. Navarro, M. Steinmetz.
%  {\it ApJ}, {\bf 554}:114, 2001.
%\bibitem{hay} The structural evolution of substructure. E. hayashi, 
%  J. F. Navarro, J. E. Taylor, J. Stadel, T. Quinn. {\it ApJ}, {\bf
%    584}:541, 2003.
\bibitem{her} N-Body realizations of compound galaxies. L. Hernquist. 
  {\it Astrophys. J. Suppl.}, {\bf 86}:389, 1993.
\bibitem{hin} Five-Year Wilkinson Microwave Anisotropy Probe (WMAP)
  Observations: Data Processing, Sky Maps, and Basic
  Results. G. Hinshaw, J. L. Weiland, R. S. Hill, N. Odegard,
  D. Larson, C. L. Bennett, J. Dunkley, B. Gold, M. R. Greason,
  N. Jarosik, E. Komatsu, M. R. Nolta, L. Page, D. N. Spergel,
  E. Wollack, M. Halpern, A. Kogut, M. Limon, S. S. Meyer,
  G. S. Tucker, E. L. Wright {\it Astrophys. J. Suppl.}, {\bf
    180}:225, 2009.
%\bibitem{updateher} Generating Equilibrium Dark Matter Halos:
%  Inadequacies of the Local Maxwellian Approximation. S. Kazantzidis,
%  J. Magorrian, B. Moore {\it ApJ}, {\bf 601}:37, 2004.
%\bibitem{kly} Galaxies in n-body simulations: Overcoming de
%  overmerging problem. A. Klypin, S. Gottl\"ober, A. V. Kravtsov,
%  A. M. Khokhlov. {\it ApJ}, {\bf 516}:530, 1999.
%\bibitem{moo01} On the destrucción and overmerging of dark halos
%  in dissipationless n-body simulations. B. Moore, N. Katz, G. Lake.
%  {\it ApJ}, {\bf 457}:455, 1996.
%\bibitem{moo02} Dark matter in Draco and the Local Group: Implications 
%for direct detection experiments. B. Moore, C. Calcáneo-Roldan, 
%J. Stadel, T. Quinn, G. Lake, S. Ghigna, F. Governato.
%{\it Phys. Rev. }, {\bf D64}:063508, 2001.  
\bibitem{nav01} The structure of cold dark matter halos. J. F. Navarro,
  C. S. Frenk, S. D. M. White. {\it ApJ}, {\bf 462}:563, 1996.
%\bibitem{nav02} A universal density profile from hierarchical
%  clustering. J. F. Navarro, C. S. Frenk, S. D. M. White. {\it ApJ},
%  {\bf 490}:493, 1997.
%\bibitem{pee} Structure of the Coma cluster of galaxies.
%  P. J. E. Peebles. {\it AJ}, {\bf 75}:13, 1970
\bibitem{gad2} The cosmological simulation code
  GADGET-2. V. Springel. {\it MNRAS}, {\bf 364}:1105, 2005.
%\bibitem{mill} The Aquarius Project: the subhalos of galactic
%halos. V. Springel, J. Wang, M. Vogelsberger, A. Ludlow,
%A. Jenkins, A. Helmi, J. F. Navarro, C. S. Frenk, S.
%D. M. White {\it MNRAS}, {\bf 391}:1685, 2008.
\bibitem{sta} PKDGRAV, comunicación privada con J. Stadel.
%\bibitem{staa}Quantifying the heart of darkness with GHALO - a
%  multi-billion particle simulation of our galactic halo. J. Stadel,
%  D. Potter, B. Moore, J. Diemand, P. Madau, M. Zemp, M. Kuhlen,
%  V. Quilis {\tt astro-ph/0808.2981v2}, 2008.
%\bibitem{tur03} Ten things everybody should know about
%  inflation. M. S. Turner {\tt astro-ph/9704062}, 1997.
\bibitem{vanK} Improved numerical modeling of clusters of galaxies.
  E. van Kampen. {\it MNRAS}, {\bf 273}:295, 1995.
%\bibitem{wei01} Adiabatic invariants in stellar dynamics 1: Basic Concepts. 
%  M. D. Weinberg. {\it AJ}, {\bf 108}:1398, 1994
%\bibitem{wei02} Adiabatic invariants in stellar dynamics 2:
%  Gravitational shocking. M. D. Weinberg. {\it AJ}, {\bf 108}:1403, 1994
%\bibitem{whi02} The dinamics of rich clusters of galaxies.
%  S. D. M. White. {\it MNRAS}, {\bf 177}:717, 1976. 
\bibitem{nbox} The Graininess of Dark Matter Haloes. M. Zemp,
J. Diemand, M. Kuhlen, P. Madau, B. Moore, D. Potter,
J. Stadel, L. Widrow.  {\it MNRAS}, {\bf 394}:641, 2009. 
\end{thebibliography}



\end{document}
